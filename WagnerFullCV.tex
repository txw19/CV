\documentclass[10pt]{article}

\usepackage{graphicx, color}
\usepackage{lineno}
\usepackage[margin=0.75in]{geometry}
\usepackage{setspace} 
\usepackage{sectsty}
% For tables
\usepackage{booktabs}
\usepackage{multicol}
\usepackage{tabularx}
%\usepackage{hyperref}
\pagenumbering{gobble}


\usepackage[colorlinks]{hyperref}
\hypersetup{
     colorlinks   = true,
     urlcolor  = blue,
     linkcolor=red,
     citecolor=Gray
}

\usepackage{etaremune} % reverse numbering
\usepackage{enumitem}

% Remove ugly url borders
%\usepackage{xcolor}
%\hypersetup{
%    colorlinks,
%    linkcolor={red!50!black},
%    citecolor={blue!50!black},
%    urlcolor={blue!80!black}
%}


\begin{document}

\centerline {\bf{\Large Tyler Wagner}}
\vspace{8pt}
\begin{flushleft}
Assistant Unit Leader and Adjunct Professor of Fisheries Ecology \hfill \href{mailto:txw19@psu.edu}{txw19@psu.edu}\\
PA Cooperative Fish \& Wildlife Research Unit, USGS \hfill Ph: 814-865-6592\\
 The Pennsylvania State University  \hfill Fax: 814-863-4710\\
 Department of Ecosystem Science \& Management \hfill  \url{http://www.wagneraqualab.com/}\\
University Park, PA 16802\\

%\hfill \url{http://www.coopunits.org/Pennsylvania/People/Tyler_Wagner/index.html}

\vspace{8pt}

\centerline {\bf{EDUCATION}}

\vspace{5pt}
{\sl Ph.D.}, 
Fisheries \& Wildlife, 
Michigan State University, E. Lansing, MI \hfill 2006 \\ 
\vspace{5pt}
{\sl M.S.}, Fisheries Resources,
University of Idaho, Moscow, ID \hfill 2000

\vspace{5pt}

{\sl B.S.}, Fisheries Resources,
University of Idaho, Moscow, ID \hfill 1999

\vspace{8pt}
\centerline {\bf{POSITIONS HELD}}
\vspace{5pt}
Assistant Unit Leader, PA Cooperative Fish \& Wildlife Research Unit, U.S. Geological Survey \hfill 2008 -- Present \\
Adjunct Professor of Fisheries Ecology, Pennsylvania State University  \hfill 2015 -- Present \\
Adjunct Assistant Professor, Department of Fisheries \& Wildlife, Michigan State University \hfill 2015 -- Present \\
Adjunct Associate Professor of Fisheries Ecology, Pennsylvania State University  \hfill 2012 -- 2015 \\
Adjunct Assistant Professor of Fisheries Ecology, Pennsylvania State University  \hfill 2008 -- 2012 \\
Postdoctoral Researcher, Quantitative Fisheries Center, Michigan State University \hfill 2006 -- 2008 \\
%\vspace{5pt}

\vspace{8pt}
\centerline {\bf{PUBLICATIONS ({\small * = graduate student, $\dagger$ = postdoc $\ddagger$ = undergraduate, $\star$ = co-leads })}}
\vspace{5pt}
\emph{2020}
\begin{etaremune}[start=104]

\item Li$^\dagger$, Y., V.S. Blazer, L.R. Iwanowicz, M.K. Schall$^\dagger$, K. Smalling, D. Tillitt, and \textbf{T. Wagner}. Accepted. Ecological risk assessment of environmental stress and bioactive chemicals to riverine fish populations: an individual-based model of smallmouth bass \textit{Micropterus dolomieu}. Ecological Modelling.

\item White$^\dagger$, S.L. and \textbf{T. Wagner}. Accepted. Behavior at short temporal scales drives dispersal dynamics and survival in a metapopulation of brook trout (Salvelinus fontinalis). Freshwater Biology.

\item Liang$^\dagger$, Z., P.A. Soranno, \textbf{T. Wagner}. 2020. The role of phosphorus and nitrogen on chlorophyll \textit{a}: evidence from hundreds of lakes. Water Research 185:116236.

\item Maynard-Bean, E. E. M. Kaye, \textbf{T. Wagner}, E.P. Burkhart. 2020. Citizen scientists record novel leaf phenology of invasive shrubs in eastern U.S. forests. Biological Invasions.	

\item \textbf{Wagner, T.}, G.J.A. Hansen, E. Schliep, B. Bethke, A. Honsey, P. Jacobson, B.C. Kline$^\ddagger$, and S.L. White$^\dagger$. 2020. Improved understanding and prediction of freshwater fish communities through the use of joint species distribution models. Canadian Journal of Fisheries and Aquatic Sciences 77: 1540-1551.

\item Schall, M.K., G.D. Smith, V.S. Blazer, H.L. Walsh, Y. Li, and \textbf{T. Wagner}. 2020. A fishery after the decline: The Susquehanna River Smallmouth Bass story. Fisheries.	
	
\item Stachelek, J. W. Weng, C.C. Carey, A.R. Kemanian, K.M. Cobourn, \textbf{T. Wagner}, K.C. Weathers, P.A. Soranno. 2020. Granular measures of agricultural land-use influence lake nitrogen and phosphorus differently at macroscales. Ecological Applications 30, e02123..

\item Brennan, J.C., R.W. Gale, D.A. Alvarez, J.K. Leet, Y. Li, \textbf{T. Wagner}, D.E. Tillitt. 2020. Factors affecting sampling strategies for design of an effects-directed analysis for endocrine-active chemicals. Environmental Toxicology and Chemistry 39:1309-1324.

\item McClure$^*$, C.M., K.L. Smalling, V.S. Blazer, A.J. Sperry, M.K. Schall, D.W. Kolpin, P.J. Phillips, M.L. Hladik, and \textbf{T. Wagner}. 2020. Spatiotemporal variation in occurrence and co-occurrence of pesticides, hormones, and other organic contaminants in rivers in the Chesapeake Bay Watershed, United States. Science of The Total Environment 728:138765.

\item  White$^\dagger$, S.L., E.M. Hanks, and \textbf{T. Wagner}. 2020. A novel quantitative framework for riverscape genetics. Ecological Applications 30:e02147.

\item White$^\dagger$, S., D. DeMario$^*$, L. Iwanowicz, V. Blazer, and \textbf{T. Wagner}. 2020. Tissue distribution and immunomodulation in channel catfish (Ictalurus punctatus) following dietary exposure to polychlorinated biphenyl Aroclors and food deprivation. International Journal of Environmental Research and Public Health 17(4)1228.

\item Soranno, P.A., K.S. Cheruvelil, B. Liu, Q. Wang, P-N. Tan, J. Zhou, K.B.S. King, I.M. McCullough, J. Stachelek, M. Bartley, C.T. Filstrup, E.M. Hanks, J-F. Lapierre, N.R. Lottig, E.M. Schliep, \textbf{T. Wagner}, K.E. Webster. 2020. Ecological prediction at macroscales using big data: Does sampling design matter? Ecological Applications 30:e02123. 

\item White$^\dagger$, S., E. Faulk, C. Tzilkowski, A.S. Weber, M. Marshall, and \textbf{T. Wagner}. 2020. Predicting fish species richness and habitat relationships using Bayesian hierarchical multispecies occupancy models. Canadian Journal of Fisheries and Aquatic Sciences 77:602-610.
	
\end{etaremune}

\vspace{5pt}
\emph{2019}
\begin{etaremune}[start=91]


\item Hansen, G.J.A., T.D. Ahrenstorff, B.J. Bethke, J. Dumke, J. Hirsch, K.E. Kovalenko, J.F. LeDuc, R.P. Maki, H. Rantala, and \textbf{T. Wagner}. Accepted. Walleye growth declines following zebra mussel and Bythotrephes invasion. Biological Invasions.

\item \textbf{Wagner, T.$^\star$}, N.R. Lottig$^\star$, M.L. Bartley, E.M. Hanks, E.M. Schliep, N.B. Wikle, K.B.S. King, I. McCullough, J. Stachelek, K.S. Cheruvelil, C.T. Filstrup, J.F. Lapierre, B. Liu, P.A. Soranno, P-N. Tan, Q. Wang, K. Webster, and J. Zhou. 2019. Increasing accuracy of lake nutrient predictions in thousands of lakes by leveraging water clarity data. Limnology and Oceanography Letters doi: 10.1002/lol2.10134. 

\item Bartley, M.L., E.M. Hanks, E.M. Schliep, P.A. Soranno, and \textbf{T. Wagner}. 2019. Identifying and characterizing extrapolation in multivariate response data. PLOSE ONE 14(12): e0225715.

\item Soranno$^\star$, P.A., \textbf{T. Wagner$^\star$}, S.M. Collins, J-F Lapierre, N.R. Lottig, and S.M. Oliver. 2019. Spatial and temporal variation of ecosystem properties at macroscales. Ecology Letters 22:1587-1598.

\item White$^*$, S.L., B.C. Kline$^\ddagger$, N.P. Hitt, and\textbf{ T. Wagner}. 2019. Individual behaviour and resource use of thermally stressed brook trout Salvelinus fontinalis portend the conservation potential of thermal refugia. Journal of Fish Biology 95:1061-1071. (featured on the cover and ``Between the JFB Covers")

\item Schall$^*$, M.K., T. Wertz, G.D. Smith, V.S. Blazer, and \textbf{T. Wagner}. 2019. Movement dynamics of smallmouth bass (\textit{Micropterus dolomieu}) in a large river-tributary system. Fisheries Management and Ecology 26:590-599.	

\item Collins, S.M., S. Yuan, P-T. Tan, S.K. Oliver, J.F. Lapierre, K.S. Cheruvelil, C.E. Fergus, N.K. Skaff, J.S. Stachelek, \textbf{T. Wagner}, P.A. Soranno. 2019. Winter precipitation and summer temperature predict lake water quality at macroscales. Water Resources Research 55:2708-2721.
	
\item Pregler, K.C., R.D. Hanks, E. Childress, N.P. Hitt, D.J. Hocking, B.H. Letcher, \textbf{T. Wagner}, and Y. Kanno. 2019. State-space analysis of power to detect regional brook trout population trends over time. Canadian Journal of Fisheries and Aquatic Sciences 76:2145-2155.

\item Li, Y. and \textbf{T. Wagner}. Does incorporating gear selectivity during macroscale investigations of fish growth reduce size-selective sampling bias in parameter estimates? 2019. Canadian Journal of Fisheries and Aquatic Sciences 76:2089-2101.

\item Midway, S.R., \textbf{T. Wagner}, and G.H. Burgess. 2019. Trends in global shark attacks. PLoS ONE
14(2): e0211049.

\end{etaremune}

\emph{2018}
\begin{etaremune}[start=81]
\item Schall$^*$, M.K., V.S. Blazer, H.L. Walsh, G.D. Smith, T. Wertz, and \textbf{T. Wagner}. 2018. Spatial and temporal variability of myxozoan parasite, Myxobolus. inornatus, prevalence in young of the year smallmouth bass in the Susquehanna River Basin, Pennsylvania. Journal of Fish Diseases 41:1689-1700.

\item Li$^\dagger$, Y., V.S. Blazer, and \textbf{T. Wagner}. 2018. Quantifying population-level effects of water temperature, flow velocity and chemical-induced reproduction depression: a simulation study with smallmouth bass. Ecological Modeling 384:63-74.

\item {\bf Wagner, T.} and E.M. Schliep. 2018. Combining nutrient, productivity, and landscape-based regressions improves predictions of lake nutrients and provides insight into nutrient coupling at macroscales. Limnology and Oceanography 63:2372-2383.

\item White$^*$, S.L., W.L. Miller, S.A. Dowell, M.L. Bartron, and {\bf T. Wagner}.  2018. Limited hatchery introgression into wild brook trout (Salvelinus fontinalis) populations despite reoccurring stocking. Evolutionary Applications 11:1567-1581.

\item Oliver, S.K., C.E. Fergus, N.K. Skaff, {\bf T. Wagner}, P-N. Tan, K.S. Cheruvelil, and P.A. Soranno. 2018. Strategies for effective collaborative manuscript development in interdisciplinary science teams. Ecosphere (4):e02206.10.1002/ecs2.2206.

\item Stow, C.A., K.E. Webster, {\bf T. Wagner}, N. Lottig, P.A. Soranno, C. YoonKynug. 2018. Small values in big data: the continuing need for appropriate metadata. Ecological Informatics 45:26-30. 

\item Massie$^\ddagger$, D.L., G.D. Smith, T.F. Bonvechio, A.J. Bunch. D.O. Lucchesi, {\bf T. Wagner}. 2018. Spatial variability and macroscale drivers of growth for native and introduced Flathead Catfish populations. Transactions of the American Fisheries Society 147:554-565.

\item DeWeber, J.T. and {\bf T. Wagner}. 2018. Probabilistic measures of climate change vulnerability, adaptation action benefits, and related uncertainty from maximum temperature metric selection. Global Change Biology 24:2735-2748.

\item Schall$^*$, M.K., V.S. Blazer, R.M. Lorantas, G.D. Smith, J.E. Mullican, B.J. Keplinger, and {\bf T. Wagner}. 2018. Quantifying temporal trends in fisheries abundance using Bayesian dynamic linear models: A case-study of riverine smallmouth bass populations. North American Journal of Fisheries Management 38:493-501.

\item Peoples, B., S.R. Midway, J.T. DeWeber, and {\bf T. Wagner}. 2018. Catchment scale determinants of nonindigenous minnow richness in the eastern United States. Ecology of Freshwater Fish 27:138-145.

\item Li$^\dagger$, Y., {\bf T. Wagner}, Y. Jiao, R. Lorantas, and C.A Murphy. 2018. Evaluating spatial and temporal variability in growth and mortality for recreational fisheries with limited catch data. Canadian Journal of Fisheries and Aquatic Sciences 75:1436-1452.

\item Filstrup, C.T., {\bf T. Wagner}, S.K. Oliver, C.A. Stow, K.E. Webster, E.H. Stanley, and J.A. Downing. 2018. Evidence for regional nitrogen stress on chlorophyll a in lakes across large landscape and climate gradients. Limnology and Oceanography 63:324-339.

\item Hansen, G.J.A., S.R. Midway, {\bf T. Wagner}. 2018. Walleye recruitment is less resilient to warming water temperatures in lakes with abundant largemouth bass populations. Canadian Journal of Fisheries and Aquatic Sciences 75:106-115.

\end{etaremune}
\emph{2017}
\begin{etaremune}[start=68]

\item Lottig, N.R., P.-N. Tan, {\bf T. Wagner}, K.S. Cheruvelil, P.A. Soranno, E.H. Stanley, C.E. Scott, C.A. Stow, and S. Yuan. 2017. Macroscale patterns of synchrony identify complex relationships among spatial and temporal ecosystem drivers. Ecosphere 8(12).

\item Soranno, P.A., L.C. Bacon, M. Beauchene, K.E. Bednar, E.G. Bissell, C.K. Boudreau, M.G. Boyer, M.T. Bremigan, S.R. Carpenter, J.W. Carr \ldots  {\bf T. Wagner} \ldots  and 70 co-authors. 2017. LAGOS-NE: A multi-scaled geospatial temporal database of lake ecological context and water quality for thousands of U.S. Lakes. GigaScience 6:1-22.

\item Yuan, S., J. Zhou, P-N., Tan, E. Fergus, {\bf T. Wagner}, and P.A. Soranno. Accepted. Multi-Level Multi-Task Learning for Nested Geospatial Data. The IEEE International Conference on Data Mining series (ICDM).

\item Oliver, S.k., S.M. Collins, P.A. Soranno, {\bf T. Wagner}, E.H. Stanley, J.R. Jones, C.A. Stow, and N.R. Lottig. 2017. Unexpected stasis in a changing world: Lake nutrient and chlorophyll trends since 1990. Global Change Biology 23:5455–5467.

\item Schall$^*$, M.K., M.L. Bartron, T. Wertz, J. Niles, V.S. Blazer, and {\bf T. Wagner}. 2017. Evaluation of genetic population structure of Smallmouth Bass in the Susquehanna River Basin, PA. North American Journal of Fisheries Management 37:729-740.

\item Grossman, G.D., R.F. Carline, and {\bf T. Wagner}. Brown trout (\emph{Salmo trutta}) in Spruce Creek Pennsylvania: a quarter-century perspective. 2017. Freshwater Biology 62:1143-1154.

\item Collins, S.M., S.K. Oliver, J.F. Lapierre, E.H. Stanley, J.R. Jones, {\bf T. Wagner}. and P.A. Soranno. 2017. Lake nutrient stoichiometry is less predictable than nutrient concentrations at regional and sub-continental scales. Ecological Applications 27:1529-1540.

\item Vidal, T. E., B. J. Irwin, {\bf T. Wagner}, L. G. Rudstam, J. R. Jackson, and J. R. Bence. 2017. Using Variance Structure to Quantify Responses to Perturbation in Fish Catches. Transactions of the American Fisheries Society 146:584-593. \emph{Featured Paper} and \emph{Robert L. Kendall Award for the Best Paper in the Transactions of the American Fisheries Society for 2017}

\item Sweka, J.A., L.A. Davis, {\bf T. Wagner}. 2017. Fall and winter survival of brook and brown trout in a North-Central Pennsylvania watershed. Transactions of the American Fisheries Society 146:744-752.

\item Cheruvelil, S. K., S. Yuan, K.E. Webster, P-N Tan, J.F. Lapierre, S.M. Collins, C.E. Fergus, C.E. Scott, E.N. Henry, P.A. Soranno, C.T. Filstrup, {\bf T. Wagner}. 2017. Creating multi-themed ecological regions for macrosystems ecology: Testing a flexible, repeatable, and accessible clustering method. Ecology and Evolution 7:3046-3058.

\item {\bf Wagner}, T. J.B. Whittier, J.T. DeWeber, S.R. Midway, and C.P. Paukert. 2017. Annual changes in seasonal river water temperatures in the eastern and western United States. Water 9(2), 90; doi:10.3390/w9020090

\item Midway, S.R., C.T. Hasler, {\bf T. Wagner}, and C.D. Suski. 2017. Predation of freshwater fish in elevated carbon dioxide environments. Marine and Freshwater Resources 68:1585-1592.

\item White$^*$, S.L., {\bf T. Wagner}, C. Gowand, and V.A. Braithwaite. 2017. Can personality predict individual differences in brook trout spatial learning ability? Behavioural Processes 141:220-228. 

\end{etaremune}
\emph{2016}
\begin{etaremune}[start=55]

\item Fergus, C.E., A.O. Finley, P.A. Soranno, {\bf T. Wagner}. 2016. Spatial variation in nutrient and water color effects on lake chlorophyll at macroscales. PLoS ONE 11(10):e0164592.doi:10.1371/journal.pone.0164592.

\item {\bf Wagner, T}., C.E. Fergus, C.A. Stow, K.S. Cheruvelil, and P.A. Soranno. 2016. The statistical power to detect cross-scale interactions at macroscales. Ecosphere 7(7):e01417.

\item Oliver, S.K., P.A. Soranno, C.E. Fergus, {\bf T. Wagner}, L.A. Winslow, C.E. Scott, K.E. Webster, J.A. Downing, and E.A. Stanley. 2016. Prediction of lake depth across a 17-state region in the U.S. Inland Waters 6:314-324.

\item Davis$^*$, L.A. and {\bf T. Wagner}. 2016. Scale-dependent seasonal pool habitat use of sympatric wild brook trout and brown trout populations. Transactions of the American Fisheries Society 145:888-902.

\item {\bf Wagner, T.}, S.R. Midway, T. Vidal, B.J. Irwin, and J.R. Jackson. 2016. Detecting unusual temporal patterns in fisheries time series data. Transactions of the American Fisheries Society 145:786-794.

\item Midway$^\dagger$, S.R., {\bf T. Wagner}, J.D. Zydlewski, B.J. Irwin, and C.P. Paukert. 2016. Transboundary fisheries science: meeting the challenges of inland fisheries management in the 21st century. Fisheries 41:536-546.

\item Soranno, P.A., K.S. Cheruvelil, {\bf T. Wagner}, K.E. Webster, and M.T. Bremigan. 2016. Effects of land use on lake nutrients: the importance of scale, hydrologic connectivity, and region. PLoS ONE  10(8): e0135454.

\end{etaremune}
\emph{2015}
\begin{etaremune}[start=48]
% 2015
\item Midway$^\dagger$, S.R. and {\bf T. Wagner}. 2015. The first description of oarfish \emph{Regalecus glesne} (Regalecidae) ageing structures. Journal of Applied Ichthyology 1-4.

\item Soranno, P.A., E.G. Bissell, K.S. Cheruvelil, S.M. Collins, C.E. Fregus, C.T. Filstrup, J-F. Lapierre, N.R. Lottig, S.K. Oliver, C.E. Scott, N.J. Smith, S. Stopyak, S. Yuan, M.T. Bremigan, J.A. Downing, C. Gries, E.N. Henry, N.K. Skaff, E.H. Stanley, C.A. Stow, P-N. Tan, {\bf T. Wagner}, and K.E. Webster. 2015. Building a multi-scaled geospatial temporal ecology database from disparate data sources: fostering open science and data reuse. GigaScience 4:28.

\item Smith$^*$, L.A., {\bf T. Wagner}, M.L. Bartron. 2015. Spatial and temporal movement dynamics of brook \emph{Salvelinus fontinalis} and brown trout \emph{Salmo trutta}. Environmental Biology of Fishes 98:2049-2065.

\item Midway$^\dagger$, S. R., {\bf T. Wagner}, S. Arnott, P. Biondo, F. Martinez-Andrade, and T. Wadsworth. 2015. Spatial and temporal variability in growth of southern flounder (\emph{Paralichthys lethostigma}). Fisheries Research 167:323-332.

\item DeWeber$^*$, J.T. and {\bf T. Wagner}. 2015. Translating climate change effects into everyday language: an example of more driving and less angling. Fisheries 40:395-398.

\item Midway$^\dagger$, S., {\bf T. Wagner}, B. H. Tracy, G. M. Hogue, and W.C. Starnes. 2015. Evaluating changes in stream fish species richness over a 50-year time-period within a landscape context. Environmental Biology of Fishes 98:1295-1309.

\item DeWeber$^*$, J.T and {\bf T. Wagner}. 2015. Predicting brook trout occurrence in stream reaches throughout their native range in the eastern United States. Transactions of the American Fisheries Society 144:11-24. 

\end{etaremune}
\emph{2014}
\begin{etaremune}[start=41]
% 2014
\item {\bf Wagner, T}., and S. R. Midway$^\dagger$. 2014. Modeling spatially varying landscape change points in species occurrence thresholds. Ecosphere 5(11):145. http://dx.doi.org/10.1890/ES14-00288.1 

\item DePasquale, C., {\bf T. Wagner}, G.A. Archard, B. Ferguson, and V.A. Braithwaite. 2014. Learning rate and temperament in a high predation risk environment. Oecologia 176:661-667. 

\item Kepler$^*$, M.V., {\bf T. Wagner}, and J.A. Sweka. 2014. Comparative bioenergetics modeling of two lake trout morphotypes. Transactions of the American Fisheries Society 143:1592-1604.

\item Perles, S.J., {\bf T. Wagner}, B.J. Irwin, D.R. Manning, K.K. Callahan, and M.R. Marshall. 2014. Evaluation of a regional monitoring program's statistical power to detect temporal trends in forest health indicators. Environmental Management 54:641-655.

\item Midway$^\dagger$, S.M., {\bf T. Wagner}, and B. Tracy. 2014. A hierarchical community occurrence model for North Carolina stream fish. Transactions of the American Fisheries Society 143:1348-1357.

\item Filstrup, C.T., {\bf T. Wagner}, P.A. Soranno, E.H. Stanley, C.A. Stow, K.E. Webster, and J.A. Downing. 2014. Regional variability among nonlinear chlorophyll-phosphorus relationships in lakes. Limnology and Oceanography 59:1691-1703. 

\item {\bf Wagner, T}., J.T. Deweber$^*$, J. Detar, D. Kristine, and J.A. Sweka. 2014. Spatial and temporal dynamics in brook trout density: implications for population monitoring. North American Journal of Fisheries Management 34:258-269.

\item Lottig, N.R., {\bf T. Wagner}, E.H. Norton, K. Spence Cheruvelil, K.E. Webster, et al. 2014. Long-term citizen-collected data reveal geographical patterns and temporal trends in lake water clarity. PLoS ONE 9(4): e95769. 

\item DeWeber$^*$, J.T., and {\bf T. Wagner}. 2014. A regional neural network model for predicting mean daily river water temperature. Journal of Hydrology 517:187-200.

\item DeWeber$^*$, J.T., Y., Tsang, D.M. Krueger, J.B. Whittier, {\bf T. Wagner}, D.M. Infante, and G. Whelan. 2014. Importance of understanding landscape biases in USGS gage locations: implications and solutions for managers. Fisheries 39:155-163. 

\item Detar, J., D. Kristine, {\bf T. Wagner}, and T. Greene. 2014. Evaluation of catch-and-release regulations on brook trout in Pennsylvania streams. North American Journal of Fisheries Management 34:49-56.

\item Levy, O., B.A. Ball, B. Bond-Lamberty, K.S. Cheruvelil,  A.O. Finley, N. Lottig, S.W. Punyasena, J. Xiao, J. Zhou, L.B. Buckley, C.T. Filstrup, T. Keitt, J.R. Kellner, A.K. Knapp, A.D. Richardson, D. Tcheng, M. Toomey, R. Vargas, J.W. Voordeckers, {\bf T.  Wagner}, J.W. Williams. 2014. Approaches to advance scientific understanding of macrosystems ecology. Frontiers in Ecology and the Environment 12:15-23. 

\item Soranno, P.A., K. Spence Cheruvelil, E. Bissell, M. Tate-Bremigan, J.A. Downing, C.E. Fergus, C. Filstrup, N.R. Lottig, E.N. Henry, E.H. Stanley, C.A. Stow, P.N. Tan, {\bf T. Wagner}, and K.E. Webster. 2014. Cross-scale interactions: quantifying multi-scaled cause-effect relationships in macrosystems. Frontiers in Ecology and the Environment 12:65-73. 

\end{etaremune}
\emph{2013}
\begin{etaremune}[start=28]
% 2013
\item Mollenhauer$^*$, R., {\bf T. Wagner}, M.V. Kepler, and J.A. Sweka. 2013. Fall and early winter movement and habitat use of wild brook trout. Transactions of the American Fisheries Society 142:1167-1178.

\item {\bf Wagner, T}., B.J. Irwin, J.R. Bence, and D.B. Hayes. 2013. Detecting temporal trends in freshwater fisheries surveys: statistical power and the important linkages between management questions and monitoring objectives. Fisheries 38:309-319.

\item {\bf Wagner, T}., J.T. Deweber$^*$, J. Detar, and J.A. Sweka. 2013. Landscape-scale evaluation of asymmetric interactions between brown trout and brook trout using two-species occupancy models. Transactions of the American Fisheries Society 142:353-361.

\item Irwin, B.J., {\bf T. Wagner}, J.R. Bence, M.V. Kepler$^*$, W. Liu, and D.B. Hayes. 2013. Estimating spatial and temporal components of variation for fisheries count data using negative binomial mixed models. Transactions of the American Fisheries Society 142:171-183.

\end{etaremune}
\emph{2012}
\begin{etaremune}[start=24]
% 2012
\item Sweka, J.A., {\bf T. Wagner}, J. Detar, and D. Kristine. 2012. Combining field data with computer simulations to determine a representative reach for brook trout assessment. Journal of Fish and Wildlife Management 3:209-222.

\item Rennie, M.D., M.P. Ebener, and {\bf T. Wagner}. 2012. Can migration mitigate the effects of ecosystem change? Patterns of dispersal, energy acquisition and allocation in Great Lakes lake whitefish (\emph{Coregonus clupeaformis}). Proceedings of the 10th Annual Coregonid Symposium. Advances in Limnology 63:455-476. 

\end{etaremune}
\emph{2011}
\begin{etaremune}[start=22]
% 2011
\item {\bf Wagner, T}., D.R. Diefenbach, A.S. Norton, and S.A. Christensen. 2011. Using multilevel models to quantify heterogeneity in resource selection. Journal of Wildlife Management 75:1788-1796. 

\item Soranno, P.A., {\bf T. Wagner}, S. Martin, L. McLean, L. Novitski,C. Provence, and A. Rober. 2011. Quantifying regional reference conditions for freshwater ecosystem management: A comparison of approaches and future research needs. Lake and Reservoir Management 27:138-148. 

\item {\bf Wagner, T}., P.A. Soranno, K.E. Webster, and K. Spence Cheruvelil. 2011. Landscape drivers of regional variation in the relationship between total phosphorus and chlorophyll in lakes. Freshwater Biology 56:1811-1824.

\item {\bf Wagner, T}., and J.A. Sweka. 2011.Evaluation of hypotheses for describing temporal trends in Atlantic salmon parr densities in Northeast U.S. Rivers. North American Journal of Fisheries Management 31:340–351. 

\end{etaremune}
\emph{2010}
\begin{etaremune}[start=18]
% 2010
\item Soranno, P.A., K. Spence Cheruvelil, K.E. Webster, M.T. Bremigan, {\bf T. Wagner}, and C.A. Stow. 2010. Freshwater ecosystem classification for landscape-scale management. BioScience 60:440-454.

\item Miksis-Olds, J. L., and {\bf T. Wagner}. 2010. Behavioral response of manatees to variations in environmental sound levels. Marine Mammal Science 27:130-148.

\item {\bf Wagner, T}., and 7 coauthors. 2010. Spatial and temporal dynamics of lake whitefish (\emph{Coregonus clupeaformis}) health measures: linking individual-based indicators to a management-relevant endpoint. Journal of Great Lakes Research 36:121-134.

\end{etaremune}
\emph{2009}
\begin{etaremune}[start=15]
% 2009
\item {\bf Wagner, T}., C.S. Vandergoot, and J. Tyson. 2009. Evaluating the power to detect temporal trends in fishery-independent surveys: a case study based on gillnets set in the Ohio waters of Lake Erie for walleye. North American Journal of Fisheries Management 29:805-816.

\end{etaremune}
\emph{2008}
\begin{etaremune}[start=14]
% 2008
\item {\bf Wagner, T}., M.E. Benbow, T.O. Brenden, J. Qi, and R.C. Johnson. 2008. Buruli ulcer disease prevalence in Benin, West Africa: associations with land use/cover and the identification of disease clusters. International Journal of Health Geographics 7:25.

\item Brenden, T.O., {\bf T. Wagner}, and B.R. Murphy. 2008. Generalized linear mixed modeling as a framework for analyzing proportional size structure index data. North American Journal of Fisheries Management 28:1233-1242.

\item {\bf Wagner, T}., M.E. Benbow, M. Burns, R.W. Merritt, J. Qi, P.L.C. Small, and R.C. Johnson. 2008. A landscape-based model for predicting \emph{Mycobacterium ulcerans} infection (Buruli ulcer) presence in Benin, West Africa. EcoHealth 5:69-79.

\item Spence Cheruvelil, K., P.A. Soranno, M.T. Bremigan, {\bf T. Wagner}, and S.L. Martin. 2008. Grouping lakes for water quality assessment and monitoring: the roles of regionalization and spatial scale. Environmental Management 41:425-440. 

\item {\bf Wagner, T}., P.A. Soranno, K. Spence Cheruvelil, B. Renwick, K. Webster, P. Vaux, and R. Abbitt. 2008. Quantifying sample biases of inland lake sampling programs in relation to lake surface area and land use/cover. Environmental Monitoring and Assessment 141:131-147.

\end{etaremune}
\emph{2007}
\begin{etaremune}[start=9]
% 2007
\item {\bf Wagner, T}., M.T. Bremigan, K. Spence Cheruvelil, P.A. Soranno, N.N. Nate, J.E. and Breck. 2007. A multilevel modeling approach to assessing regional and local landscape features for lake classification and assessment of fish growth rates. Environmental Monitoring and Assessment 130:437-454.

\item {\bf Wagner, T}., J.R. Bence, M.T. Bremigan, D.B. Hayes, and M.J. Wilberg. 2007. Regional trends in fish mean length at age: components of variance and the power to detect trends. Canadian Journal of Fisheries and Aquatic Sciences 64:968-978.

\end{etaremune}
\emph{2006}
\begin{etaremune}[start=7]
% 2006
\item {\bf Wagner, T}., A.K. Jubar, and M.T. Bremigan. 2006. Can habitat alteration and spring angling explain black bass nest distribution and success? Transactions of the American Fisheries Society 135:843-852.

\item {\bf Wagner, T}., D.B. Hayes, and M.T. Bremigan. 2006. Accounting for multilevel data structures in fisheries data using mixed models. Fisheries 31:180-187.

\item Congleton, J. L., and {\bf T. Wagner}. 2006. Blood-chemistry indicators of nutritional status in juvenile salmonids. Journal of Fish Biology 69:473-790.

\end{etaremune}
\emph{2005}
\begin{etaremune}[start=4]
%2005
\item Zabel, R.W., {\bf T. Wagner}, J.L. Congleton, S.G. Smith, and S.G. Williams. 2005. Survival and selection of migrating salmon from capture-recapture models with individual traits. Ecological Applications 15:1427-1439. 

\end{etaremune}
\emph{2004}
\begin{etaremune}[start=3]
% 2004
\item {\bf Wagner, T}., and J.L. Congleton. 2004. Blood-chemistry correlates of nutritional condition, tissue damage, and stress in migrating juvenile chinook salmon (\emph{Oncorhynchus tshawytscha}). Canadian Journal of Fisheries and Aquatic Sciences 61:1066-1074.

\item {\bf Wagner, T}., J.L. Congleton, and D.M. Marsh. 2004. Smolt-to-adult return rates of juvenile chinook salmon transported through the Snake-Columbia River hydropower system, USA, in relation to densities of co-transported juvenile steelhead. Fisheries Research 68:259-270.

\end{etaremune}
\emph{2002}
\begin{etaremune}[start=1]
% 2002
\item {\bf Wagner, T}. and C.M. Falter. 2002. Response of an aquatic macrophyte community to fluctuating water levels in an oligotrophic lake. Lake and Reservoir Management 18:52-65.
\end{etaremune}

\vspace{8pt}
% PRESENTATIONS
\centerline {\bf{PRESENTATIONS ({\small $\bullet$ = undergraduate; * = graduate student, $\dagger$ = postdoc, $\ddagger$ = invited})}}
\vspace{5pt}
\begin{etaremune}

\item Krause, K., K. Maloney, \textbf{T. Wagner}. 2020. Analyzing stream fish communities of the Chesapeake Bay watershed: A joint species distribution modeling approach. Annual Meeting of the American Fisheries Society. 

\item McClure$^*$, C., K. Smalling, V.Blazer, and \textbf{T. Wagner}. 2020. Maternal sourcing of contaminants from ovary to juvenile Smallmouth Bass in the Chesapeake Bay Watershed. Pennsylvania Chapter of the American Fisheries Society.

\item Massie$^*$, D.L., Hansen, G., Li. Y., and \textbf{T. Wagner}. 2020. Do lake-specific characteristics mediate the temporal relationship between Walleye growth and warming water temperatures? Pennsylvania Chapter of the American Fisheries Society.

\item Li, Y. and \textbf{T. Wagner}. 2019. Ecological risk assessment of environmental stress and bioactive chemicals to riverine fish populations: an individual-based model of smallmouth bass. American Geophysical Union Annual Conference.

\item \textbf{Wagner$^\ddagger$, T.} 2019 Putting together pieces of the puzzle: contaminants, pathogens, and smallmouth bass population dynamics in the Susquehanna River Basin. Water Insights Seminar, Pennsylvania State University. 

\item Stachelek, J., C.C. Carey, K.M. Cobourn, S.M. Collins, A.R. Kemanian, \textbf{T. Wagner}, K.C. Weathers, W. Weng, and P.A. Soranno. 2019. Analysis of 500 lake catchments reveals the relationship between crop type, fertilizer and manure inputs and lake nutrient concentrations. 2019. Ecological Society of America Annual Meeting.

\item White$^*$, S.L., E.M. Hanks, and \textbf{T. Wagner}. 2019. A novel quantitative framework for riverscape genetics highlights the importance of mainstem channels for brook trout population connectivity. Annual Meeting of the American Fisheries Society.

\item Massie$^*$, D.L., G. Hansen, Y. Li. and \textbf{T. Wagner}. 2019. Do lake-specific characteristics mediate the temporal relationship between Walleye growth and warming water temperature? Annual Meeting of the American Fisheries Society, Reno, NV.

\item McClure, C. K. Smalling, V. Blazer, and \textbf{T. Wagner}. 2019. The spatiotemporal dynamics of contaminants in streams of the Chesapeake Bay Watershed. Annual Meeting of the American Fisheries Society.

\item Smith, G.D., M.K. Schall, V.S. Blazer, H.L. Walsh, and \textbf{T. Wagner}. 2019. The role of disease in altering the population structure of Smallmouth Bass in the Susquehanna River Basin. International Association for Great Lakes Research (IAGLR). 

\item Schall, M.K., V.S. Blazer, H.L. Walsh, G. Smith, T. Wertz, and \textbf{T. Wagner}. 2019. Quantifying spatial variability in young of year smallmouth bass disease infections in the Chesapeake Bay Watershed. International Association for Great Lakes Research (IAGLR), Brockport, NY.

\item Massie$^*$, D.L., G.D. Smith, T.F. Bonvechio, A.J. Bunch. D.O. Lucchesi, {\bf T. Wagner}. 2018. Spatial variability and macroscale drivers of growth for native and introduced Flathead Catfish populations. Annual Meeting of The American Fisheries Society, Atlantic City, New Jersey.

\item White$^*$, S.L., W.L. Miller, S.A. Dowell, M.L. Bartron, and {\bf T. Wagner}. 2018. Limited hatchery introgression into wild brook trout populations despite reoccurring stocking. Annual Meeting of The American Fisheries Society, Atlantic City, New Jersey.

\item Sweka, J.A. and {\bf T. Wagner}. Effects of stream discharge on young-of-year brook trout density in North Central Pennsylvania. Annual Meeting of the American Fisheries Society, Atlantic City, NJ.

\item Kline$^\bullet$, B., S.L. White, N. Hitt, and {\bf T. Wagner}. Resource use by brook trout (Salvelinus fontinalis) in a thermally complex environment. Annual Meeting of The American Fisheries Society, Atlantic City, New Jersey.

\item {\bf Wagner, T.}, E. Schliep, G. Hansen, B. Bethke, and P. Jacobson. 2018. C'mon, everyone?s doing it: joint distribution models for studying spatiotemporal dynamics of fish communities and their habitat . Annual Meeting of the American Fisheries Society, Atlantic City, NJ.

\item Cheruvelil, K.S., {\bf T. Wagner}, K. Webster, K. King, A. Poisson. 2018. Macroscale patterns and drivers of phosphorus and chlorophyll in shallow lakes. Association for the Sciences of Limnology and Oceanography, Summer Meeting, Victoria, BC, Canada.

\item Filstrup, C.T., J-F. Lapierre, S.K. Oliver, P.A. Soranno, {\bf T. Wagner}, and J.A. Downing. 2018. Spatiotemporal patterns in extreme lake chlorophyll concentrations at the sub-continental scale. Association for the Sciences of Limnology and Oceanography, Summer Meeting, Victoria, BC, Canada.

\item Soranno, P.A., {\bf T. Wagner}, S.M. Collins, J-F. Lapierre, N.R. Lottig, S.K. Oliver. 2018. Spatial variation exceeds temporal variation in lake ecosystem properties at macroscales. Association for the Sciences of Limnology and Oceanography, Summer Meeting, Victoria, BC, Canada.

\item Schall$^\dagger$, M.K., V.S. Blazer, H.L. Walsh, G. Smith, R. Lorantas, T. Wertz, and {bf T. Wagner}. 2018. Investigating occurrence of disease characteristics and trends in smallmouth bass abundance in rivers within the Chesapeake Bay Watershed. Chesapeake Research \& Modeling Symposium, Annapolis, MD.

\item Collins, S.M., S. Yuan, P.-N. Tan, S.K. Oliver, J.F. Lapierre, K.S. Cheruvelil, E. Fergus, N.K. Skaff, J. Stachelek, {\bf T. Wagner}, P.A. Soranno. 2018.c. Society for Freshwater Science, Detroit MI.

\item Thompson$^*$, T.J., V. Blazer, A. Sperry, M. Briggs, and {\bf T. Wagner}. 2018. Groundwater as a source of emerging contaminants to streams of the Chesapeake Bay Watershed. Society for Freshwater Science, Detroit MI.

\item White$^*$, S.L. and {\bf T. Wagner}. 2018. With Connectivity Comes Challenges: Brook Trout Metapopulation Dynamics Reveal Unique Management Challenges. Society for Freshwater Science, Detroit MI.

\item White$^*$, S. W. Miller, S. Dowell, M. Bartron, and {\bf T. Wagner}. 2017. Reproduction of stocked and wild brook trout in Loyalsock Creek  Susquehanna River Symposium, Bucknell University.

\item Thompson$^*$, T., {\bf T. Wagner}, V. Blazer, A. Sperry, and M.A. Briggs. 2017. Groundwater as a source of emerging contaminants. Susquehanna River Symposium, Bucknell University.

\item Kline$^\bullet$, B. S. White$^*$, N. Hitt, and {\bf T. Wagner}. 2017. Personality predicts success at using thermal refugia in brook trout (Salvelinus fontinalis). Susquehanna River Symposium, Bucknell University.

\item {\bf Wagner, T}$^\ddagger$. 2017. "Movers" and "Stayers": understanding brook trout movement for conservation and management. Pennsylvania Wild Trout Summit. The Future of Wild Trout in Pennsylvania.

\item Thompson$^*$, T.J., {\bf T. Wagner}, V.S. Blazer, P. Phillips, M. Briggs, and A. Sperry. 2017. Groundwater as a source of emerging contaminants in the Chesapeake Bay. U.S. Geological Survey Chesapeake Bay Meeting.

\item Schall$^*$, M.K., V.S. Blazer, H. Walsh, T. Wertz, G. Smith, and {\bf T. Wagner}. 2017. Investigating myxozoan parasite prevalence in young of the year smallmouth bass in the Susquehanna River basin, 2013-2016. U.S. Geological Survey Chesapeake Bay Meeting.

\item White$^*$, S., S. Dowell, M. Bartron, and {\bf T. Wagner}. 2017. Where do all the fish go? Combining multiple measures of fish movement to gain insights into brook trout population connectivity. The Annual Meeting of the American Fisheries Society, Tampa FL.

\item Noring, A.M., G.G. Sass, S.R. Midway, J.A. VanDeHey, J.K. Raabe, D.A. Isermann, J.M. Kampa, T.P. Parks, J. Lyons, M.J. Jennings, G.J.A. Hansen, and {\bf T. Wagner}. 2017. Effects of Cisco on Walleye Growth Trajectories in Northern Wisconsin Lakes. 13th International Coregonid Symposium.

\item Collins, S., K.S. Cheruvelil, E. Fergus, J.F. Lapierre, S. Oliver, J. Skaff, P.A. Soranno, P-N. Tan, {\bf T. Wagner}, S. Yuan. 2017. Which measures of climate are the best predictors of lake water quality at sub-continental scales? Ecological Society of America Annual Meeting, Portland, OR. 

\item White$^*$, S., S. Dowell, M. Bartron, and {\bf T. Wagner. 2017}. Where do all the fish go? Combining multiple measures of fish movement to gain insights into brook trout population connectivity. Wild Trout Symposium, West Yellowstone, MT.

\item Schall$^*$, M.K., V.S. Blazer, H.L. Walsh, G. Smith, T. Wertz, and {\bf T. Wagner}. 2017. Evaluating differences in field observations and histological prevalence of myxozoan parasites in young-of-year smallmouth bass in the Susquehanna River Basin, PA. AFS FHS Annual Meeting.

\item Filstrup, C.T.,  {\bf T. Wagner}, C.A. Stow, S.K. Oliver, E.H. Stanley, K.E. Webster, J.A. Downing. 2016. Nitrogen stress effects on lake phytoplankton vary by region based on land use. Association for the Sciences of Limnology and Oceanography. 

\item Lottig, N.R., P-N. Tan, K.S. Cheruvelil, C.E. Scott, E.H. Stanley, P.A. Soranno, C.A. Stow, {\bf T. Wagner}, S. Yuan. 2016. Long-term patterns and drivers of water quality at sub-continental spatial scales. Association for the Sciences of Limnology and Oceanography. 

\item Massie$^\bullet$, D., G. Smith, {\bf T. Wagner}. 2016. Comparing relative abundance and population characteristics of Flathead Catfish across a range of establishment levels at the Susquehanna River. Susquehanna River Symposium, Bucknell University. 

\item Rhoads$^\bullet$, S., S. White$^*$, {\bf T. Wagner}, and J. Niles. Movers and stayers: what factors influence brook trout movement? Susquehanna River Symposium, Bucknell University.

\item Schall$^*$, M., {\bf T. Wagner}, M. Bartron, V.S. Blazer and J. Niles. 2016. Movement dynamics and population genetics of smallmouth bass in the Susquehanna River Basin. Susquehanna River Symposium, Bucknell University.

\item White$^*$, S., L. Iwanowicz, {\bf T. Wagner}. 2016. Stream temperature and stress protein regulation in Brook Trout. Susquehanna River Symposium, Bucknell University. 

\item Li$^\dagger$, Y.,  {\bf T. Wagner}, Y. Jiao, V. Blazer, D. Tillit, P. Phillips, and D. Kolpin.2016. Risk assessment of endocrine disrupting compounds on fish and wildlife populations in Chesapeake Bay watershed: a simulation study with smallmouth bass. USGS Chesapeake Bay Workshop.

\item Schall$^*$, MK., M.L. Bartron, T. Wertz, J. Niles, V.S. Blazer, {\bf T. Wagner}. 2016. Investigating movement dynamics and population genetics of smallmouth bass in the Susquehanna River Basin. USGS Chesapeake Bay Workshop.

Thompson$^*$, T., M. Schall$^*$, {\bf T. Wagner}, V. Blazer, A. Sperry and J. Niles. 2016. An investigation into the role of groundwater as a point source of emerging contaminants to smallmouth bass in the Susquehanna River basin. USGS Chesapeake Bay Workshop.

 \item {\bf Wagner$^\ddagger$, T}. 2016. Risk assessment of contaminants to fish and wildlife populations in the Chesapeake Bay Watershed: a simulation study with smallmouth bass. PA Department of Environmental Protection, Statewide Annual Biologist Meeting.

\item Lapierre, J-F., S.M. Collins, D. Seekell, P.A. Soranno, K.S. Cheruvelil, P-N. Tan, C.E. Fergus, N. Skaff,  {\bf T. Wagner}, M.T. Bremigan. 2016. Aligning spatial scales improves understanding of biogeochemical relationships between climate, landscape, and limnological properties. Association for the Sciences of Limnology and Oceanography.

\item Oliver, S.K., S. Collins, K.S. Cheruvelil, P.A. Soranno, E.H. Stanley, J-F. Lapierre, N. Lottig, and  {\bf T. Wagner}. 2016. Long-term change in lake nutrient concentrations: where are we now? Association for the Sciences of Limnology and Oceanography. 

\item Collins, S.M., S.K. Oliver, J-F. Lapierre, E.H. Stanley, J. Jones,  {\bf T. Wagner}, P.A. Soranno. 2016. What drives lake nutrients at continental scales, and why is it so hard to predict nutrient ratios? Association for the Sciences of Limnology and Oceanography. 

\item Li$^\ddagger$, Y. and  {\bf T. Wagner}. 2016. Assessing the impacts of endocrine disrupting compounds on fish population dynamics: a case study of smallmouth bass in Pennsylvania, USA. National Conference on Ecosystem Restoration. 

\item Sweka, J.A., L. Davis, and {\bf T. Wagner}. 2016. Survival of Brook and Brown Trout through the Spawning Season in a North Central Pennsylvania Watershed. 146th AFS Annual Meeting in Kansas City, Missouri.

\item Midway, S.R., B. Peoples, J.T. DeWeber, and {\bf T. Wagner}. 2016. Native congener richness, not abiotic factors, predicts cyprinid introductions. Annual Meeting 146th AFS Annual Meeting in Kansas City, Missouri. 


% 2015
\item Ikis, D., E. Post, and {\bf T. Wagner}. 2015. Bird Occupancy Dynamics in Alaskan Wetlands: Why different size ponds matter?  Ecological Society of America Annual Meeting. 

\item Lapierre, J-F., S.M. Collins, C. Scott, K.S. Cheruvelil, P-N. Tan, M.T. Bremigan, {\bf T. Wagner}, and P.A. Soranno. 2015. The role of spatial structure in determining the strength of the relationships among climate, landscape and limnological properties. Ecological Society of America Annual Meeting.\\

\item Scott, C.E., C.E. Fergus, S.M. Collins, J-F. Lapierre, N.R. Lottig, C.T. Filstrup, N. Skaff, E.H. Stanley, P-N. Tan, {\bf T. Wagner}, P.A. Soranno, and K.S. Cheruvelil. 2015. Understanding the response of lake water quality at macroscales to measures of regional and global climate. Ecological Society of America Annual Meeting.

\item Fergus, C.E., A.O. Finley, P.A. Soranno, {\bf T. Wagner}. 2015. Examining the nutrient-color paradigm across macroscales: Multivariate spatial relationships among lake phosphorus, water color, and chlorophyll.  Ecological Society of America Annual Meeting.\\

\item Cheruvelil, K.S., S. Yuan, S. Collins, C.E. Fergus, C. Filstrup, E. Norton Henry, J-F. Lapierre, C. Scott, P. Soranno, P-N. Tan, {\bf T. Wagner}, K. Webster. 2015. Including the freshwater landscape in a multi-themed regionalization system to capture macroscale patterns. Ecological Society of America Annual Meeting.

\item White$^*$, S.L., C. Gowan, {\bf T. Wagner}, V.A. Braithwaite. 2015. Boldness impairs spatial learning ability in brook trout. American Fisheries Society Annual Meeting, Portland, Oregon.

\item DeWeber, T. and {\bf T. Wagner}. 2015. Predicting mean daily river water temperature to identify brook trout habitat. American Fisheries Society Annual Meeting, Portland, Oregon.

\item DeWeber, T. and {\bf T. Wagner}. 2015. Is that the best metric for predicting climate change effects on brook trout? American Fisheries Society Annual Meeting, Portland, Oregon.

\item Schall$^*$, M.K., V.S. Blazer, {\bf T. Wagner, T}. Wertz, and G. Smith.  2015. Investigation of contaminants and disease characteristics of young of the year smallmouth bass in the Susquehanna River basin, PA. AFS-Fish Health Section Annual Meeting, Ithaca, NY. 

\item Lorantas, R., {\bf T. Wagner}, D. Arnold, J. Detar, M. Kaufmann, K. Kuhn, R. Lorson, R. Wnuk, and A. Woomer 2015. Characterizing survival of smallmouth bass from Age 0 to Age 1 in Pennsylvania river sections using electrofishing survey gear catch rate and regression residuals.  The Annual Northeast Fish \& Wildlife Conference. 

\item Schall$^\ddagger$, M.K.,  {\bf T. Wagner}, V. Blazer, and T. Wertz. 2015. Movement dynamics of smallmouth bass in the Susquehanna River basin. Pennsylvania-Ohio Joint American Fisheries Society Annual Meeting.

\item Filstrup, C.T., S.K. Oliver, E.H. Stanley, C.A. Stow, {\bf T. Wagner}, K.E. Webster, and J.A. Downing. 2015. Regional land use influences nitrogen subsidy-stress effects on lake phytoplankton. Association for the Sciences of Limnology and Oceanography 2015 Aquatic Sciences Meeting, Granada, Spain. 


% 2014
\item Oliver SK, Stanley EH, Cheruvelil KS, Downing J, Fergus CE, Soranno PA, {\bf Wagner T}, Webster K, Winslow L. 2014. Prediction and patterns of lake depth across a 17-state region in the U.S. Joint Aquatic Sciences Meeting. Portland, OR.

\item Kepler$^*$, M.V., V. Blazer, {\bf T. Wagner}, H. Walsh, G. Smith. 2014. Evaluation of potential disease causing agents in young of the year smallmouth bass in the Chesapeake Bay Watershed. International Symposium on Aquatic Animal Health. Portland, OR. 

\item Irwin$^\ddagger$, B. J., T. Vidal, {\bf T. Wagner}, J. R. Bence, J. R. Jackson, L. G. Rudstam, and W. W. Fetzer. 2014. Shifting variance structure as an indicator of large-scale ecological change. Annual meeting of the Ecological Society of America.

\item Faulk$^*$, E. and {\bf T. Wagner}. 2014. Stream fish communities in the Delaware Water Gap National Recreation Area: A multi-species occupancy approach. American Fisheries Society, 144th Annual Meeting.

\item Vidal, T., C. Jansch, B. J. Irwin, {\bf T. Wagner}, J. R. Bence, J. R. Jackson, L. G. Rudstam, and W. W. Fetzer. 2014. Using variance structure as statistical indicators of large scale ecological change. 144th Annual meeting of the American Fisheries Society.  

\item Filstrup, C.T., S. Oliver, E.H. Stanley, C.A. Stow, {\bf T. Wagner}, K.E. Webster, and J.A. Downing. 2014. Subsidy-stress effects of nitrogen on phytoplankton biomass. Joint Aquatic Sciences Meeting. 

\item Lottig, N.R., P-N Tan, K.S. Cheruvelil, C.E. Scott, P.A. Soranno, C.A. Stow, S. Yuan, and {\bf T. Wagner}. 2014. Taxonomy of change: using cluster analysis to identify temporal patterns in limnological data. Joint Aquatic Sciences Meeting. 

\item Scott, C.E., C.E. Fergus, N.R. Lottig, C.T. Filstrup, {\bf T. Wagner}, E.H. Stanley, and P.A. Soranno. 2014. Which global and regional climate metrics at macroscales best describe lake water quality responses to climate change? Joint Aquatic Sciences Meeting. 

\item Midway$^\dagger$, S. M., S. Wayne, G. Hogue, {\bf T. Wagner}, and B. Tracy. 2014. Evaluating changes in stream fish species richness over a 50-year time period within a landscape context. Southern Division of the American Fisheries Society Spring Meeting. 

\item Deweber$^*$, J.T. and {\bf T. Wagner}. 2014. A model for predicting daily river water temperature in the Northeast and its utility for management. 70th Annual Northeast Fish \& Wildlife Conference, Portland, Maine.

\item Deweber, J.T. and  {\bf T. Wagner}. 2014. The future of brook trout in changing climate and landscape. 70th Annual Northeast Fish \& Wildlife Conference, Portland, Maine.


% 2013
\item {\bf Wagner$^\ddagger$, T. 2013}. Brook Trout ecology and management: land use, climate and transboundary population monitoring. University of Georgia. 

\item Smith$^*$, L., and {\bf T. Wagner}. 2013. Seasonal movement patterns and habitat use of the eastern brook trout in north-central Pennsylvania. Annual Meeting of the American Fisheries Society. 

\item Deweber$^*$, J. T. and {\bf T. Wagner}. 2013. Climate and land use change implications for native brook trout management. American Fisheries Society Pennsylvania Chapter Fall Technical Meeting.

\item Smith$^*$, L. and {\bf T. Wagner}. 2013. Seasonal movement patterns and habitat use of the eastern brook trout (\emph{Salvelinus fontinalis}) and brown trout (\emph{Salmo trutta}) in north central Pennsylvania. American Fisheries Society Pennsylvania Chapter Fall Technical Meeting. 

\item {\bf Wagner, T}., J.T. Deweber$^*$, J. Detar, D. Kristine, and J.A. Sweka. 2013. Spatial and Temporal Dynamics in Brook Trout Density: Implications for Population Monitoring. American Fisheries Society Pennsylvania Chapter Fall Technical Meeting.

\item Smith$^*$, L. and {\bf T. Wagner}. 2013. Seasonal movement patterns and habitat use of the eastern brook trout in north-central Pennsylvania. Annual Meeting of the American Fisheries Society.

\item Kepler$^*$, M.V.,  {\bf T. Wagner}, J. A. Sweka. 2013. Comparative bioenergetics modeling of two representative strains of lean and humper lake trout (\emph{Salvenlinus namaycush}) morphotypes. 69th Annual Northeast Fish \& Wildlife Conference, Saratoga Springs, New York. 

\item Soranno$^\ddagger$, P.A., K.S. Cheruvelil, E. Bissell, M.T. Bremigan, J.A. Downing, C.E. Fergus, C.T. Filstrup, N.R. Lottig, E. Norton-Henry, E.H. Stanley, C. Stow, P-N. Tan, {\bf T. Wagner} and K. Webster. 2013. A conceptual framework for understanding multi-scaled cause-effect relationships between terrestrial and aquatic ecosystems. Ecological Society of America, 98th Annual Meeting.

\item Filstrup, C. T., {\bf T. Wagner}, P.A. Soranno, E.H. Hill, C.A. Stow. 2013. Regional variability in non-linear chlorophyll response to total phosphorus enrichment in lakes. ASLO 2013 Aquatic Sciences Meeting,New Orleans, LA.

\item Irwin, B. J., and {\bf T. Wagner}. 2013. Shifting variance structure as a potential indicator of fish-population responses to large-scale perturbation. Southern Division of the American Fisheries Society Annual Meeting.

\item Irwin, B. J., and {\bf T. Wagner}. 2013. Using mixed models to quantify variability in fish populations. GA Chapter of the American Fisheries Society, Jekyll Island, GA.


% 2012
\item Filstrup, C.T., P.A. Soranno, E.H. Stanley, C.A. Stow, {\bf T. Wagner}, K.E. Webster, and J.A. Downing. 2012. Chlorophyll a response to total phosphorus enrichment in lakes differs by spatial scales and among regions: implications for developing water quality criteria at the state level. North American Lake Management Society 32nd International Symposium, Madison, Wisconsin. 

\item Kristine, D., J. Detar, R. Lorantas, R. Lorson,  {\bf T. Wagner}, D. DeMario. 2012. Assessment of experimental panfish regulations on select Pennsylvania impoundments. American Fisheries Society Annual Meeting. Twin Cities, MN.

\item Salvesen$^*$, K., M. Bartron, and {\bf T. Wagner. 2012}. Genetic assessment of the Klondike lake trout strain: Comparison of wild and hatchery strains. American Fisheries Society Annual Meeting. 

\item Deweber$^*$, T. and  {\bf T. Wagner}. 2012. Linking climate and land use projections to stream habitat. American Fisheries Society Annual Meeting. Twin Cities, MN. 

\item {\bf Wagner$^\ddagger$, T}. 2012. Hierarchical Models. Michigan State University.

 \item {\bf Wagner, T}., B.J. Irwin, J.R. Bence, W. Lui, D.B. Hayes. 2012. Developing expectations for detecting temporal change in freshwater fisheries surveys. 68th Annual Northeast Fish and Wildlife Conference. Charleston, West Virginia. 

\item Detar, J., D. Kristine, T. Greene, and {\bf T. Wagner}. 2012. Evaluation of catch and release regulations for wild brook trout in Pennsylvania streams. East Coast Trout Management and Culture Workshop V, Frostburg, MD.

\item Lottig, N., E. Stanley, P. Hanson, and {\bf T. Wagner}. 2012. Long-term changes in lake chemistry related to sulfate deposition and climate. North American Lake Management Society 32nd International Symposium.

\item Lottig, N., {\bf T. Wagner}, E. Norton, K. Cheruvelil, C.A. Stow, J.A. Downing, and K.E. Webster. 2012. Non-monotonic trends in citizen-based regional lake water clarity. North American Lake Management Society 32nd International Symposium.


% 2011
\item DeMario$^*$, D.A., L.R. Iwanowicz, {\bf T. Wagner}, M.V. Kepler, and D.C. Honeyfield. 2011. Effects of dietary restriction on PCB congener dynamics and their association with health in channel catfish. American Fisheries Society Annual Meeting. 

\item {\bf Wagner, T}., B.J. Irwin, J.R. Bence, W. Liu, and D.B. Hayes. 2011. The role of variance components and survey design in detecting trends in recreational fisheries monitoring data. 6th World Recreational Fishing Conference. 

\item Cheruvelil K.S., P.A. Soranno, K.E. Webster, M.T. Bremigan, {\bf T.Wagner}, and C.A. Stow. 2010. Landscape limnology: integrating freshwater, terrestrial, and human landscapes for ecological understanding, natural resource management and conservation. ASLO/NABS Summer Meeting.

\item Sweka, J.A., {\bf T. Wagner}, J. Detar. 2011. Determination of a representative reach to estimate brook trout biomass in Pennsylvania streams. American Fisheries Society 141st Annual Meeting. Seattle, WA.

\item Irwin, B.J., {\bf T. Wagner}, W. Liu, J.R. Bence, and D.B. Hayes. 2011. Developing negative binomial mixed models to partition variance in fishery-independent survey data. American Fisheries Society 141st Annual Meeting. Seattle, WA.

\item Mollenhauer$^*$, R., J. Sweka, M. Kepler, and  {\bf T. Wagner}. 2011. Seasonal movement and habitat use of wild brook trout in central Pennsylvania. Pennsylvania Chapter of AFS Spring Technical Meeting.

\item Mollenhauer$^*$, R., J. Sweka, M. Kepler, and  {\bf T. Wagner}. 2011. Habitat use of wild brook trout in central Pennsylvania. 67th Annual Northeast Fish and Wildlife Conference, Manchester, New Hampshire.

\item Sweka, J.,  {\bf T. Wagner}, and J. Detar. 2011. Determination of a representative sample reach for estimation of native brook trout biomass in Pennsylvania streams. 67th Annual Northeast Fish and Wildlife Conference, Manchester, New Hampshire.

\item Sweka$^\ddagger$, J. and  {\bf T. Wagner}. 2011. Evaluation of hypotheses for describing temporal trends in Atlantic salmon parr densities in Northeast U.S. Rivers. The Connecticut River Research Forum.

\item Deweber, J. T. and {\bf T. Wagner}. 2011. Predicted impacts of climate change on stream habitat and brook trout populations in the eastern United States. 6th World Recreational Fishing Conference, Berlin, Germany.


% 2010
\item Diefenbach$^\ddagger$, D. R., {\bf T. Wagner}, R. D. Brubaker, E. H. Just. 2010. Managing white-tailed deer to restore and maintain plant species diversity. Annual Conference of The Wildlife Society, Snowbird, Utah. 

\item Diefenbach, D. R.,  {\bf T. Wagner}, S. A. Christensen, A. S. Norton. 2010. Using multilevel models to quantify heterogeneity in resource selection. Annual Conference of The Wildlife Society, Snowbird, Utah.

\item Detar, J., D. Kristine, R.T. Greene, R. Weber, and  {\bf T. Wagner}. 2010. Evaluation of brook trout enhancement regulations in Northcentral Pennsylvania. Annual Meeting of the American Fisheries Society. 

\item Sweka, J.A. and  {\bf T. Wagner}. 2010. Evaluation of hypotheses for explaining temporal trends in Atlantic salmon parr densities in Northeast U.S. Rivers. Annual Meeting of the American Fisheries Society.

\item Kepler$^*$, M.V.,  {\bf T. Wagner}, B.J. Irwin, J.R. Bence, D.B. Hayes, and N.P. Lester. 2010. Spatial and temporal variation in Great Lakes percid catch-per-effort data. Annual Meeting of the American Fisheries Society. 

\item Cheruvelil, K.S. P.A. Soranno, K.E. Webster, M.T. Bremigan, {\bf T. Wagner}, C.A. Stow. 2010. Landscape limnology: integrating freshwater, terrestrial, and human landscapes for ecological understanding, natural resource management and conservation. ASLO/NABS Summer Meeting.

\item Rennie, M. D., M.P. Ebener, and {\bf T. Wagner}. 2010. Can migration mitigate the effects of ecosystem change? Patterns of dispersal, energy acquisition and allocation in Great Lakes lake whitefish (\emph{Coregonus clupeaformis}). International Association for Great Lakes Research. Toronto, Canada. 

\item Lorantas, R.M., {\bf T. Wagner}, D.A. Miko, D.A. Arnold, J. Detar, M.L. Kaufman, K. Kuhn, R. Lorson, and R.T. Wnuk. 2010. Evaluation of riverine smallmouth bass recruitment indices on Pennsylvania Rivers. Annual Meeting of the American Fisheries Society. 


% 2009
\item {\bf Wagner, T}. K. Spence Cheruvelil, P.A. Soranno, K.E. Webster. 2009. A hierarchical Bayesian approach to modeling regional variation in total phosphorus – chlorophyll a relationships. North American Lake Management Society.

\item Cheruvelil, K.S., P.A. Soranno, K.E. Webster, M.T. Bremigan, {\bf T. Wagner}, C.A. Stow. 2009. Freshwater ecosystem classification for landscape-scale management. 57th Annual Meeting of the North American Benthological Society. 

\item DeMario$^*$ D., {\bf T. Wagner}, D. Miko, R. Lorantas, R. Lorson, J. Detar, D. Kristine, J. Weigle. 2009. Evaluation of panfish enhancement regulations in Pennsylvania lakes. 65th Northeastern Fish and Wildlife Conference.

\item {\bf Wagner T}., P.A. Soranno, M.T. Bremigan, K.S. Cheruvelil, K.E. Webster, C.A. Stow. 2009. Freshwater ecosystem classification for landscape-scale management. 65th Northeastern Fish and Wildlife Conference.


% 2008
\item Bremigan$^\ddagger$, M., P. Soranno, M. Gonzalez,  B. Bunnell, K. Arend, W. Renwick, R. Stein, and M. Vanni, K.S. Cheruvelil, K. Webster, C. Stow, and {\bf T. Wagner}. 2008. Linking foodweb and landscape models: hydrogeomorphic mediation of land-use effects on reservoirs. 69th Midwest Fisheries and Wildlife Conference.

\item {\bf Wagner, T}., Jones, M. L., Ebener, M. P., Arts, M. P., Brenden, T. O., Honeyfield, D. C., Wright, G. M., Faisal, M. 2008. Spatial and temporal dynamics of lake whitefish health indicators: linking individual-based indicators to a management-relevant endpoint. 69th Midwest Fisheries and Wildlife Conference.

\item Arts M.T., A. Blukacz, R. Claramunt, M. Ebener, M. Faisal, J. Fitzsimmons, D. Honeyfield, J. Hoyle, T. Johnson, M. Jones, R.E. Kinnunen, M.A. Koops, T. Mezek, A.M. Muir, A. Richards, T.M. Sutton, {\bf T. Wagner}, and G. Wright. 2008. Fatty acid profiles of lake whitefish (\emph{Coregonus clupeaformis}) in the ever-changing Great Lakes. American Society of Limnology and Oceanography Aquatic Sciences Meeting.

\item {\bf Wagner, T}., B. Irwin, J. R. Bence, D. B. Hayes, and N. Lester. 2008. Spatial and temporal components of variation in Great Lakes fish populations: implications for management and conservation.  Annual meeting of the Great Lakes Fishery Commission’s Fishery Research Board.


% 2007
\item Arts M.T., A. Blukacz, R. Claramunt, M. Ebener, M. Faisal, J. Fitzsimmons, D. Honeyfield, J. Hoyle, T. Johnson, M. Jones, R.E. Kinnunen, M.A. Koops, T. Mezek, A.M. Muir, A. Richards, T.M. Sutton, {\bf T. Wagner}, and G. Wright. 2007. Spatial and temporal patterns in fatty acid profiles of lake whitefish (\emph{Coregonus clupeaformis}) in the Great Lakes in relation to fish condition and inferred diet. Plenary talk. Societas Internationalis Limnologia (SIL) 30th Congress. 

\item Benbow$^\ddagger$, M.E., {\bf T. Wagner}, T. Brenden, P. Suykerbuyk, M. Burns, R.C. Johnson, M. McIntosh, R. Kimbirauskas, R. Kolar, H. Williamson, RW. Merritt, J. Qi, P.L.C. Small, D. Boakye, C. Quaye, and F. Portaels. 2007. New frontiers into the ecology of Buruli ulcer disease – an update on using satellite imagery to quantify landscape-ecology linkages with disease occurrence in West Africa. World Health Organization: Annual meeting of the Global Buruli Ulcer Initiative.


% 2006
\item {\bf Wagner$^\ddagger$, T}., M.E. Benbow, M. Burns, R.E. Kolar, R.W. Merritt, J. Qi, and P.L.C. Small. 2006. A landscape-based model for predicting \emph{Mycobacterium ulcerans} infection (Buruli ulcer) presence/absence in Benin, West Africa. Ecology of Infectious Diseases – PI Network Meeting in conjunction with the American Society of Tropical Medicine and Hygiene annual meeting. 

\item {\bf Wagner, T}., J.R. Bence, M.T. Bremigan, D.B. Hayes, and M.J. Wilberg. 2006. Regional trends in fish mean length at age: components of variance and the power to detect trends. Annual Meeting of the American Fisheries Society. 


% 2005
\item {\bf Wagner, T}., M.T. Bremigan, K.S. Cheruvelil, P.A. Soranno, N.A. Nate, and J.E. Breck. 2005. Comparing multiscale predictors of fish growth: towards a regional framework for fish management. American Society of Limnology and Oceanography Aquatic Sciences Meeting. 


% 2004
\item {\bf Wagner, T}., M.T. Bremigan, K.S. Cheruvelil, P.A. Soranno, N.A. Nate, and J.E. Breck. 2004. Comparing multiscale predictors of fish growth: towards a regional framework for fish management. Midwest Fish and Wildlife Conference.

\item {\bf Wagner, T}., A.K. Jubar, and M.T. Bremigan. 2004. Can habitat alteration and spring fishing explain black bass nest distribution and success? Midwest Fish and Wildlife Conference.


% 2003
\item Congleton, J., B. LaVoie, {\bf T. Wagner}, D. Jones, D. Fryer, J. Evavold, and B. Sun. 2003. Blood-chemistry correlates of nutritional condition in migrating juvenile chinook salmon. Symposium on Use of Physiology to Assist in Management of Declining Fish Stocks, Annual Meeting of American Fisheries Society.

\item {\bf Wagner, T}. and J.L. Congleton. 2003. Chemical indices in migrating juvenile chinook salmon: putting together the pieces of the puzzle. Symposium on Use of Physiology to Assist in Management of Declining Fish Stocks, Annual Meeting of American Fisheries Society.

\item Jones, D. T., {\bf T. Wagner}, and J.L. Congleton. 2003. Blood chemistry and swimming performance of fed and fasted juvenile Chinook salmon exposed to confinement stressors. 24th Annual Smolt Workshop.


% 2002
\item {\bf Wagner, T}. and J.L. Congleton. 2002. Evaluation of physiological condition of transported salmonids and effects on survival. Anadromous Fish Evaluation Program Annual Meeting.


% 2001
\item {\bf Wagner, T}. and C.M. Falter. 2001. Response of an aquatic macrophyte community to fluctuating water levels in an oligotrophic lake. Idaho Chapter American Fisheries Society.


% 2000
\item {\bf Wagner, T}. and C.M. Falter. 2000. The effects of higher winter water levels on the aquatic macrophyte community of lake Pend Oreille, Idaho, North American Lake Management Society Symposium.


\end{etaremune}

% GRANTS
\centerline {\bf{GRANTS (over \$11,400,000 in total funding)}}
\vspace{5pt}
\begin{etaremune}

\item \textit{National Science Foundation}: Scale, Space, and Time: A Unifying Approach to Aquatic Invasions. PI: Brandon Peoples (Clemson), co-PIs: Steve Midway (LSU), Julian Olden (UW), Shweta
 Singh (Purdue), Traci Birch (LSU), Matthew Hiatt (LSU), Patricia Carbajales (Clemson), Stuart. Borrett (UNCW), Senior project personnel: Gretchen Hansen (Univ. Minnesota), and Tyler Wagner. Total \$731,464; Amount to Wagner: \$0.00


\item {\sl USGS National Climate Adaptation Science Center}: Quantifying the impacts of climate change on fish growth and production to enable sustainable management of diverse inland fisheries. PI: Gretchen Hansen (Univ. Minnesota). co-PIs Olaf Jensen, University of Wisconsin-Madison; Jordan Read, USGS Water Mission Area; Craig Paukert, USGS. Cooperators and partners: Tyler Ahrenstorff, MNDNR; James Bence, Michigan State University; David Bunnell, USGS; Zachary Feiner, WDNR; Abigail Lynch, USGS National Climate Adaptation Center; Charles Madenjian, USGS; Heidi Rantala, MNDNR; Greg Sass, WIDNR; Tyler Wagner, USGS; Joe Nohner, MIDNR/MGLP.Total \$674,260; Amount to Wagner: \$0.00

\item {\sl U.S. Geological Survey}: Determining the consequences of land management actions on primary drivers influencing smallmouth bass populations; PI: Tyler Wagner. June 2020 - May 2025. Total \$575,000; Amount to Wagner: \$575,000

\item {\sl Pennsylvania Sea Grant}: Quantifying the Roles of Changing Watershed Conditions and Biotic Interactions in Structuring Pennsylvania Stream Fish Communities; PI: Tyler Wagner, co-PIs: Tim Wertz (PA DEP), Matt Shank (SRBC), Megan Schall (PSU), Geoff Smith and Doug Fischer (PA Fish \& Boat Commission). February 2020 - January 2022. Total \$100,000; Amount to Wagner: \$100,000

\item  {\sl USGS National Climate Adaptation Science Center}: Fish habitat restoration to promote adaptation: resilience of sport fish in lakes of the Upper Midwest. PI: Gretchen Hansen (Univ. Minnesota), co-PIs: Tyler Wagner, Jordan Read (USGS); Erin Schliep (Univ. of Missouri); Zach Feiner (WI DNR); Catherine Hein (WI DNR); Pete Jacobson (MN DNR); Joe Nohner (Midwest Glacial Lakes Partnership and MI DNR); Samantha Oliver (USGS); Kevin Wehrly (MI DNR); Abigail Lynch (USGS National Climate Adaptation Science Center). September 2019 - August 2023. Total \$495,955; Amount to Wagner: \$0.00

\item  {\sl Pennsylvania Sea Grant}: Diet composition of invasive Flathead Catfish in the Susquehanna River Basin: quantifying impacts on native and migratory fishes and recreational fisheries; PI: Megan Schall (PSU), co-PIs: Tyler Wagner, Geoff Smith (PA Fish \& Boat Commission) and Julian Avery (PSU). February 2020 - January 2022. Total \$183,841; Amount to Wagner: \$0.00

\item {\sl Pennsylvania Sea Grant}: Comparison of age and growth patterns of Flathead Catfish in invasive and native populations: A meta-analysis with implications for invasive species management in Pennsylvania; co-PIs: Tyler Wagner and Geoff Smith (PA Fish \& Boat Commission). February 2018 - January 2020; Total: \$94,869;  Amount to Wagner: \$86,369

\item {\sl  R.K. Mellon Freshwater Research Initiative}: Micro-RNA profiles of brook trout in response to thermal stress; PI: Tyler Wagner, co-PIs: Luke Iwanowicz (USGS), Shannon White (PSU). March 2017 - May 2018; Total: \$10,000; Amount to Wagner: \$10,000

\item {\sl  R.K. Mellon Freshwater Research Initiative}: Emerging contaminants in groundwater: implications for smallmouth bass health in the Susquehanna River Basin; PI: Tyler Wagner, co-PIs: Vicki Blazer (USGS). March 2017 - May 2018; Total: \$13,827; Amount to Wagner: \$13,827

\item {\sl National Science Foundation}, MacroSystems Biology and Early NEON Science: A macrosystems ecology framework for continental-scale prediction and understanding of lakes; PI: Patricia. Soranno (MSU), co-PIs: Kendra Cheruvelil (MSU), Pang-Ning Tan (MSU), Jiayu Zhou (MSU), Emily Stanley (UW), Corinna Gries (UW), Noah Lottig (UW), Tyler Wagner, Ephraim Hanks (PSU), Erin Schliep (UM). Oct 2016 - Sept 2021; Total: \$4,257,250; Amount to Wagner: \$268,876

\item {\sl U.S. Geological Survey}: Establishing a strategy for assessing risk of endocrine-disrupting compounds to aquatic and terrestrial organisms; PI: Tyler Wagner, Co-PIs (all USGS): Vicki Blazer Donald Tillett, Patrick Phillips. May 2016 - Sept 2019; Total \$408,380;  Amount to Wagner: \$408,380

\item {\sl Pennsylvania Sea Grant}: Preliminary determination of density and distribution of Flathead Catfish Pylodictis olivaris in the Susquehanna River and select tributaries; co-PIs: Tyler Wagner and Geoff Smith (PA Fish \& Boat Commission). March 2016 - December 2017; Total: \$71,072;  Amount to Wagner: \$65,322

\item {\sl  R.K. Mellon Freshwater Research Initiative}: An investigation into the role of groundwater as a point source of emerging contaminants to smallmouth bass in the Susquehanna River basin; PI: Tyler Wagner, co-PIs: Vicki Blazer (USGS), Megan Schall (PSU), Jonathan Niles (SU). April 2016 - March 2018; Total: \$19,891; Amount to Wagner: \$19,891

\item {\sl  R.K. Mellon Freshwater Research Initiative}: Population genetic structure of brook trout in the Loyalsock Creek watershed; PI: Tyler Wagner, co-PIs: Victoria Braithwaite (PSU), Shannon White (PSU), Meredith Bartron (USFWS), Jonathan Niles (SU). April 2015 - May 2016; Total: \$12,500; Amount to Wagner: \$12,500

\item {\sl  R.K. Mellon Freshwater Research Initiative}: Phenotype-specific gene expression in brook trout in the Loyalsock Creek Watershed; PI: Tyler Wagner, co-PIs: Shannon White (PSU), Victoria Braithwaite (PSU), Luke Iwanoxicz (USGS), Jonathan Niles (SU). May 2016 - March 2018; Total: \$17,500; Amount to Wagner: \$17,500

\item {\sl  R.K. Mellon Freshwater Research Initiative}: Investigation of genetic population structure of smallmouth bass in the Susquehanna River basin; PI: Tyler Wagner, co-PI: Megan Schall (PSU). April 2015 - May 2016; Total: \$13,500; Amount to Wagner: \$13,500

\item {\sl Pennsylvania Sea Grant}: Investigating the role of contaminants and parasite prevalence in the observed mortality of smallmouth bass in the Susquehanna River basin; PI: Tyler Wagner, co-PIs: Vicki Blazer (USGS), Geoff Smith (PA Fish \& Boat Commission). November 2104 - Jan 2016; Total: \$74,900; Amount to Wagner: \$74,900

\item {\sl Pennsylvania Sea Grant}: Quantifying seasonal movement dynamics and thermal habitat use of smallmouth bass in the Susquehanna River basin: implications for fish disease and fisheries management; PI: Tyler Wagner. Feb 2014 - Jan 2016; Total: \$30,000; Amount to Wagner: \$30,000

\item {\sl U.S. Geological Survey}, Pennsylvania Water Resources Research Center Small grants program: Determining how fish populations cope with rapid environmental fluctuation: A cast study in Pennsylvania streams; co-PIs: Tyler Wagner and Victoria Braithwaite (PSU), Shannon White (PSU). Mar 2015 - Feb 2016; Total: \$17,950; Amount to Wagner: \$17,950

\item {\sl U.S. Geological Survey}, Northeast Climate and Wildlife Science Center: A decision support mapper for conserving stream fish habitats of the Northeast; PI: Craig Paukert (USGS), Co-PIs: Dana Infante (MSU), Tyler Wagner, Jana Stewart (USGS), Joanna Whittier (UM). Jul 2013 - June 2016; Total: \$199,881; Amount to Wagner: \$0.00

\item {\sl National Park Service}: Fish community assessment in the Eastern Rivers and Mountains Network and integration with existing monitoring data; PI: Tyler Wagner. Jan 2013 - May 2015;  Total: \$148,615; Amount to Wagner: \$148,615

\item {\sl US Fish \& Wildlife Service}: Structured decision making for Key Deer management and recovery; Co-PIs (both USGS): Duane Diefenbach, Tyler Wagner. Oct 2013 - June 2014; Total: \$81,386; Amount to Wagner: \$0.00

\item {\sl U.S. Geological Survey}, Northeast Climate and Wildlife Science Center: Characterization of spatial and temporal variability in fishes in response to climate change; Co-PIs Tyler Wagner, Brian Irwin (USGS), and Jim Bence (MSU). July 2012 - August 2017; Total: \$149,945; Amount to Wagner: \$0.00

\item {\sl  U.S. Geological Survey},  Chesapeake Bay Priority Ecosystems Science: Linking fish health, contaminants, and population dynamics of smallmouth bass populations in the Susquehanna River, Pennsylvania; PI: Tyler Wagner. May 2013 - Dec 2018; Total: \$300,000; Amount to Wagner: \$300,000

\item {\sl  U.S. Geological Survey}: Transboundary management and conservation: linking large-scale dynamics to ecological monitoring and management; PI: Tyler Wagner, co-PIs: Brian Irwin (USGS), Joseph Zydlewski (USGS). Oct 2012 - Sept 2014; Total: \$117,427; Amount to Wagner: \$117,427

\item {\sl National Science Foundation},  MacroSystems Biology and Early NEON Science: The effects of cross-scale interactions on freshwater ecosystem state across space and time; PI: Patricia Soranno (MSU), Co-PIs: Kendra Cheruveli (MSU), Pang-Ning Tan (MSU), Tyler Wagner, Emily Stanley (UM) and others. May 2011 - May 2016; Total: \$1,310,583;  Amount to Wagner: \$0.00

\item {\sl U.S. Geological Survey}, National Climate Change \& Wildlife Science Center: Managing the nations fish habitat at multiple spatial scales in a rapidly changing climate; PI: Tyler Wagner. Dec 2009 - Mar 2013; Total: \$235,445; Amount to Wagner: \$235,445

\item {\sl  Pennsylvania Fish \& Boat Commission}: Evaluation of wild trout resources and restoration efforts in Pennsylvania; PI: Tyler Wagner. Jul 2011 - June 2015; Total: \$104,741; Amount to Wagner: \$104,741

\item  {\sl  Pennsylvania Fish \& Boat Commission}: Distributions of PCB congeners in Pennsylvania streams and fish: implications for risk management and fish health; PI: Tyler Wagner. Sept. 2009 - Dec 2011; Total: \$284,924; Amount to Wagner: \$284,924

\item {\sl U.S Fish \& Wildlife Service}: Habitat use, movement and genetic composition of lake trout in the Niagara River and Niagara Bar; PI: Tyler Wagner. Oct 2010 - June 2013; Total: \$142,292;  Amount to Wagner: \$142,292

\item  {\sl U.S Fish \& Wildlife Service}: Comparative energetics of lake trout morphotypes; PI: Tyler Wagner. Nov 2010 - June 2013; Total: \$125,050; Amount to Wagner:  \$125,050

\item  {\sl U.S Fish \& Wildlife Service}, Great Lakes Fish and Wildlife Restoration Act: Spatial and temporal components of variation in Great Lake percid populations: implications for conservation and management; PI: Tyler Wagner. co-PIs: Brian Irwin (USGS), Jim Bence (MSU). Apr 2009 - Dec 2011; Total: \$67,878; Amount to Wagner: \$67,878
\end{etaremune}

% GRANTS
%\centerline {\bf{GRANTS PENDING}}
%\vspace{5pt}
%\begin{etaremune}
%\item {\sl  National Science Foundation}, MacroSystems Biology and Early NEON Science: Predicting lake fish population and community responses to global change at macroscales, co-PIs: Tyler Wagner, Kendra Cheruvelil (MSU), Steve Midway (LSU), Noah Lottig (UW), Gretchen Hanson (MNDNR), Andrew Rypel (WIDNR); \$893,436
%\end{etaremune}

\centerline {\bf{CODE AND DATA PRODUCTS ({\small * = graduate student, $\dagger$ = postdoc $\ddagger$ = undergraduate, $\star$ = co-leads })}}
\vspace{5pt}
\begin{etaremune}
	\item Maynard-Bean, E. et al. R code and data for fitting hierarchical linear models to forest shrub phenology data in eastern forests. \url{https://doi.org/10.5281/zenodo.3939230}
	\item Thompson, T.$^*$ et al. R code and data for performing Bayesian censored regression analysis of groundwater and surface water contaminants in rivers in the Chesapeake Bay Watershed. \url{https://doi.org/10.5281/zenodo.3888674}
	\item Liang, Z.$^\dagger$, et al. R code for performing quantile regression for setting joint nutrient criteria. \url{https://doi.org/10.5281/zenodo.3956328}
	\item Li, Y.$^\dagger$ et al. R code for performing ecological risk assessment: individual-based model for smallmouth bass. \url{https://doi.org/10.5281/zenodo.3956465}
	\item Massie, D.$^*$ et al. Power Analysis: First release of framework developed in Massie et al. 2020. R scripts for performing power analysis for macroscale fish growth investigations. \url{https://doi.org/10.5281/zenodo.3610495}
	\item McClure, C.$^*$ Contaminant occurrence: Code for Bayesian hierarchical joint-contaminant model. \url{https://doi.org/10.5281/zenodo.3746590}
	\item Wagner, T. Analysis of groundwater and surface water contaminants in rivers in the Chesapeake Bay Watershed. Data and model code. \url{https://doi.org/10.5281/zenodo.3888674}
	\item Hansen, Gretchen J A; Bethke, Bethany J; Ahrenstorff, Tyler D; Dumke, Josh; Hirsch, Jodie; Kovalenko, Katya E; LeDuc, Jaime F; Maki, Ryan P; Rantala, Heidi M; Wagner, Tyler. 2019. Data and R code for analysis of walleye and yellow perch age-0 length in Minnesota's Large Lakes. Retrieved from the Data Repository for the University of Minnesota, \url{https://doi.org/10.13020/t9tr-y063}
\item Wagner T. 2019. Updated release of code and data for Wagner et al., includes new code and metadata (Version v1.2.0). \url{http://doi.org/10.5281/zenodo.3484680}
	
\item Wagner, T. 2019. txw19/Eco\_variation: Spatial and temporal variation of ecosystem properties at macroscales (Version v1.0.0). \url{http://doi.org/10.5281/zenodo.2628379}
\end{etaremune}	

%\vspace{8pt}
\centerline {\bf{AWARDS AND RECOGNITION}}

\vspace{5pt}
{\bf U.S. Geological Survey Scientific Excellence Award (2019)} - Awarded for excellence in furthering the mission of the Cooperative Research Units Program.\\
\vspace{5pt}
{\bf U.S. Geological Survey Scientific Excellence Award (2017)} - Awarded for excellence in furthering the mission of the Cooperative Research Units Program.\\
\vspace{5pt}
{\bf The Edward D. Bellis Award (2018)} - Awarded to recognize faculty members in the Intercollege Graduate Degree Programs in Ecology for outstanding contribution and dedication to educating and training graduate students in the program.\\
\vspace{5pt}
\textbf{Robert L. Kendall Award for the Best Paper in the Transactions of the American Fisheries Society (2017)}: co-author\\
\vspace{5pt}
{\bf U.S. Geological Survey Scientific Excellence Award (2015)} - Awarded for excellence in furthering the mission of the Cooperative Research Units Program.\\
\vspace{5pt}
{\bf National Fish Habitat Award for Excellence in Scientific Achievement  (2015)} - To recognize outstanding achievement in the use of science to improve fish habitat conservation. Awarded to an individual or group who has developed and/or implemented science-based tools, assessments, or methodologies that assist in the conservation of aquatic habitat. This award was given to Wagner for research on brook trout conservation and management in the eastern U.S.\\

\vspace{5pt}
% Outreach and information transfer
\vspace{6pt}
\centerline {\bf{MEDIA COVERAGE OF SCHOLARLY WORK AND INFORMATION TRANSFER}}
\vspace{5pt}

% \href{http://www.r-project.org/}{R}
\textbf{Trends in global shark attacks}\\
\begin{itemize}
	\item Manuscript on global shark attacks, published in \href{https://journals.plos.org/plosone/article?id=10.1371/journal.pone.0211049}{PLoS ONE}, was highlighted by \href{https://www.cnn.com/2019/02/27/health/shark-attack-risk-low-study/index.html}{CNN},  
	\href{https://www.usnews.com/news/health-news/articles/2019-02-27/dont-fear-shark-attacks-remain-rare
	}{US News and World Reports}, among others. The story by the 
  \href{https://www.staradvertiser.com/2019/03/01/hawaii-news/shark-attacks-up-but-risk-still-low/?HSA=02820a85be106052a2a700de45e8b4dd77e64a75}{Honolulu Star-Advertiser}
was picked up by the Associate Press, and occurred in the 
\href{https://www.nytimes.com/aponline/2019/03/02/us/ap-us-shark-bite-study.html}{New York Times}, \href{https://www.washingtonpost.com/national/health-science/study-more-sharks-bite-people-in-hawaii-but-risk-minuscule/2019/03/02/50c79b92-3d55-11e9-b10b-f05a22e75865_story.html?noredirect=on&utm_term=.2d98167dde7b}{Washington Post}, and other news outlets.
\end{itemize}

\textbf{Research on brook trout ecology in the northeastern U.S.}\\
\begin{itemize}
	\item ``Larger streams are critical for wild brook trout conservation": Penn State News link \href{https://news.psu.edu/story/621981/2020/06/03/research/larger-streams-are-critical-wild-brook-trout-conservation}{here}
	   \item Penn State brook trout researchers featured in new 'Expedition Chesapeake' film: Penn State News story -  link \href{https://news.psu.edu/story/564133/2019/03/20/research/penn-state-brook-trout-researchers-featured-new-expedition}{here}
		\item  ``Few hatchery brook trout genes present in Pennsylvania watershed wild fish'': National Science Foundation, 'News from the Field' - highlighting Penn State News story - link \href{https://www.nsf.gov/news/news_summ.jsp?cntn_id=296716}{here}
	\item ``Few hatchery brook trout genes present in Pa. watershed wild fish'': Penn State News link \href{https://news.psu.edu/story/537280/2018/09/18/research/few-hatchery-brook-trout-genes-present-pa-watershed-wild-fish}{here}
\item ``Fisheries scientists are probing ways that wild brook trout adapt to a changing world'': Pittsburgh Post-Gazette link \href{http://www.post-gazette.com/sports/outdoors/2017/12/31/Fishery-scientists-are-probing-ways-that-wild-brook-trout-adapt-to-a-changing-world/stories/201712310280}{here}\\
\item ``Science Provides a Glimpse into a Possible Future for Anglers'': U.S. Geological Survey National Climate Change and Wildlife Science Center link
\href{https://nccwsc.usgs.gov/content/science-provides-glimpse-possible-future-anglers}{here}\\
\item ``Brook trout behavior and genetics could help populations adapt to habitat change'': Pennsylvania Angler and Boater magazine link
\href{http://www.fishandboat.com/Transact/AnglerBoater/AnglerBoater2017/MayJune/Documents/2017-0506mj-10brook.pdf}{here}\\
\item ``Brook trout personality, genetics could help populations adapt to habitat change'': Penn State News link 
\href{http://news.psu.edu/story/442050/2016/12/13/brook-trout-personality-genetics-could-help-populations-adapt-habitat-change}{here}\\
\item ``A race against the clock for brook trout conservation'': The Wildlife Management Institute story is found \href{https://wildlifemanagement.institute/outdoor-news-bulletin/september-2016/race-against-clock-brook-trout-conservation}{here} and The Widlife Society coverage is found
\href{http://wildlife.org/a-race-against-the-clock-for-brook-trout-conservation/}{here}\\
\item ``Brook Trout Research (Shocking)'': Local ABC News TV affiliate coverage found \href{http://wnep.com/2016/09/25/brook-trout-research-shocking/}{here}\\
\item ``For trout fishermen, climate change will mean more driving time, less angling'': Penn State News link
\href{http://news.psu.edu/story/366131/2015/08/20/research/trout-fishermen-climate-change-will-mean-more-driving-time-less}{here}
\item ``Climate Change May Cost Fishermen More Time, Money in Search of Brook Trout'': AccuWeather coverage found here \href{http://www.accuweather.com/en/weather-news/climate-change-brook-trout-fisherman-high-cost/52151933}{here}\\

\item ``New model identifies Eastern U.S. stream sections holding wild brook trout''; Science Daily News link \href{http://www.sciencedaily.com/releases/2015/01/150106154610.htm}{here}\\
\item ``New model identifies eastern stream sections holding wild brook trout'': Penn State News link
\href{http://news.psu.edu/story/339640/2015/01/06/research/new-model-identifies-eastern-stream-sections-holding-wild-brook}{here}\\
\vspace{5pt}
\end{itemize}
\textbf{Limnology}\\
\begin{itemize}
\item ``A fresh look at fresh water: Researchers create a 50,000-lake database'': National Science Foundation \href{https://www.nsf.gov/news/news_summ.jsp?cntn_id=243391&org=NSF&from=news}{here} and R package to access database \href{https://github.com/cont-limno/LAGOSNE}{here} 
\item ``Despite Changes in Climate, Land Use and Management Practices, Lakes Stay Surprisingly Static'': UW-Madison  \href{http://blog.limnology.wisc.edu/despite-changes-in-climate-land-use-and-management-practices-lakes-stay-surprisingly-static/}{here}, Science Daily coverage  \href{https://www.sciencedaily.com/releases/2017/08/170823184403.htm}{here}, and StarTribune coverage \href{http://m.startribune.com/minnesota-lakes-holding-their-own-against-pollution/441820303/?section=local}{here}
\item ``Researchers harness 'big data' to see the big picture on lakes, nutrient cycles'': Penn State News link \href{http://news.psu.edu/story/425689/2016/09/13/research/researchers-harness-big-data-see-big-picture-lakes-nutrient-cycles}{here}\\
\item ``Citizen scientists provide clarity for lake researchers' questions'': Penn State News link
\href{http://news.psu.edu/story/314178/2014/04/30/research/citizen-scientists-provide-clarity-lake-researchers-questions}{here} \\
\item ``Clarity for lake researchers' water quality questions'': National Science Foundation press release
\href{http://www.nsf.gov/discoveries/disc_summ.jsp?cntn_id=131238&org=NSF&from=news}{here} \\
\item ``Over 60 years of citizen science observations detect trends in Midwestern lakes'': EurekAlert news release
\href{http://www.eurekalert.org/pub_releases/2014-04/p-o6y042814.php}{here} \\
\item ``Citizen scientists provide clarity for lake researchers’ big questions'': University of Wisconsin press release
\href{http://www.news.wisc.edu/22805}{here}  \\
\item ``MSU researches freshwater relationships'': Michigan State University news link \href{http://www.statenews.com/index.php/article/2010/06/msu_researches_freshwater_relationships}{here}\\
\vspace{5pt}
\end{itemize}
\textbf{Transboundary fisheries research}\\
\begin{itemize}
\item ``New Comprehensive Approach to Inland Fisheries Management'': Louisiana State University news link \href{http://www.lsu.edu/mediacenter/news/2016/09/13docs_midway_transfish.php}{here} and SciencDaily link \href{https://www.sciencedaily.com/releases/2016/09/160914130706.htm}{here}\\
\end{itemize}
\textbf{Key Deer Management}\\
\begin{itemize}
\vspace{5pt}
\item Research on the the management of the endangered Key deer was covered by Al Jazeera America and the YouTube video release is \href{http://www.youtube.com/watch?v=r6GX7tZiPFk}{here}\\
\vspace{5pt}
\end{itemize}
\textbf{Climate change and fisheries}\\
\begin{itemize}
\vspace{5pt}
\item ``From brook trout to walleyes, warming waters to play havoc with fisheries'': PSU press release \href{http://news.psu.edu/story/470451/2017/06/04/research/brook-trout-walleyes-warming-waters-play-havoc-fisheries}{here}\\
\vspace{5pt}
\end{itemize}

\vspace{8pt}
\centerline {\bf{TEACHING}}
\vspace{5pt}
\begin{itemize}
\item Quantitative Methods in Ecology, Pennsylvania State University\\
\item Advances in Ecology, Pennsylvania State University\\
\item Hierarchical Models in Ecology, Pennsylvania State University\\
\item Structured Decision Making and Adaptive Management of Natural Resources, Pennsylvania State University\\
\item Developed an on-line course introducing students to the programming language R, Michigan State University\\
\item Limnology, Michigan State University\\
\end{itemize}

\vspace{8pt}
\centerline {\bf{NON-CREDIT INSTRUCTION ({\small $\dagger$ = invited})}}
\vspace{5pt}
\begin{itemize}
\item \emph{An Introduction to R}: 2-day workshop, PA Department of Environmental Protection (September 21 - 22, 2016)\\
\item \emph{Bayesian hierarchical modeling}$^\dagger$: 3-day workshop, Virginia Institute of Marine Science (May 24 -- 26, 2016)\\
\item \emph{Hierarchical modeling of left-censored data}: 1-day workshop, Penn State Univ. (September 29, 2015)
\item \emph{Bayesian hierarchical modeling}$^\dagger$: 2-day workshop, The Univ. of Missouri (June 16 -- June 18, 2015)\\
\item \emph{Bayesian hierarchical modeling}$^\dagger$: 2-day, The Ohio State Univ. (April 22 --  April 24, 2015)\\
\item \emph{An Introduction to Bayesian Estimation and Inference Using JAGS}: 2-day workshop, Penn State Univ. (June 17 -- 18, 2013)\\
\item \emph{An Introduction to Multilevel Models and their Application in Forest, Fish, and Wildlife Management}: 2-day workshop, Penn State Univ. (September 27 -- 28, 2012)\\
\item \emph{An Introduction to R}: 2-day workshop, Penn State Univ. (August 14 -- 15, 2008)\\

\end{itemize}

\vspace{8pt}


\centerline {\bf{DIRECTED STUDENT LEARNING}}
\vspace{5pt}
\begin{tabularx}{\textwidth}{ XX }

\textbf{Major Advisor} & \textbf{Postdoctoral Researchers Advised} \\

Catherine McClure - MS (2018 - present) & Zhongyao Liang (2019-present)\\
Tyler Thompson - MS (2016 - present) &  Shannon White (2019-present)\\
Shannon White - PhD (2014 - 2019)  & Megan Schall (2018)\\
Danielle Massie - MS (2018 - present) & Yan Li (2015 - 2017)\\
Kepler, Megan - PhD (2013 - 2017) & Midway, Stephen (2013 - 2015)\\
Faulk, Evan - MS (2012 - 2015) &  \\
Salvesen, Kelley - MS (2011 - 2015) & \\
Deweber, Jefferson - PhD (2010 - 2014) & \\
Hall, Lara - MS (2013 - 2014) & \\
Smith, Lori - MS (2011 - 2013) & \\
Kepler, Megan - MS (2011 - 2013) & \\
DeMario, Devin - MS (2009 - 2013) & \\
Mollenhauer, Robert - MS (2008 - 2011) & \\
\end{tabularx}



\end{flushleft}

\centerline {\bf{UNIVERSITY SERVICE}}
\vspace{5pt}
\begin{itemize}
\item Member of the Graduate Admissions Committee for Penn State's Intercollege Graduate Degree Program in Ecology (2011-2013) 
\end{itemize}

\centerline {\bf{PROFESSIONAL SERVICE AND ENGAGEMENT}}
\vspace{5pt}
\begin{itemize}
\item Member of the Chesapeake Bay Trust Technical Review Committee (2019)
\item Editorial Board Member - Scientific Reports (2016 - 2017)
\item Secretary Treasurer - Education Section of the American Fisheries Society (2015 - 2016)
\item President, PA Chapter of the American Fisheries Society (2010 - 2011)
\end{itemize}

\centerline {\bf{TECHNICAL ADVISORY PANELS AND WORKSHOPS}}
\vspace{5pt}
\begin{itemize}
\item Member of the Chesapeake Bay Program Brook Trout Action Team to develop a management strategy for brook trout in the Chesapeake Bay
\item Member of the Chesapeake Bay brook trout technical team to provide expertise on statistical model development and application
\item Technical review panel member to review and comment on stock assessment and simulation models of management performance for the Lake Erie Percid Management Advisory Group
\item Provided technical assistance/expert opinion to Ontario Ministry of Natural Resources and Forestry, through Cambium Inc., to help guide efforts to manage and monitor both brook trout and their habitat across southern Ontario
\item Provided expert opinion on Trout Unlimited's conservation portfolio analysis for the eastern range of the brook trout
\item Facilitated Structured Decision Making workshops for the Pennsylvania Bureau of Forestry to develop a decision model for managing deer with respect to forest vegetation conditions
\end{itemize}

\centerline {\bf{CURRENT MEMBERSHIPS IN PROFESSIONAL SOCIETIES}}
\vspace{5pt}
\begin{itemize}
\item Ecological Society of America
\item American Fisheries Society
\item American Society of Limnology \& Oceanography
\item North American Lake Management Society
\end{itemize}



\end{document}
